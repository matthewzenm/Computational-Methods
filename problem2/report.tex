\documentclass{ctexart}

\usepackage{amsmath}
\usepackage{amsthm}
\usepackage{amssymb}
\usepackage{geometry}
\usepackage[shortlabels]{enumitem}
\usepackage{float}
\usepackage[ruled,linesnumbered]{algorithm2e}

\DeclareMathOperator{\sgn}{sgn}

\title{计算方法第二次上机作业}
\author{曾梦辰\thanks{学号: 202011999050}}
\date{\today}

\begin{document}
\maketitle

\section{问题}
已知以下两个方程均有一个实根, 分别使用二分法, Newton法, 割线法求近似根, 精确到小数点后$8$位, 然后对Newton法采用Aitken加速.
\begin{align*}
    &\sin{x}=6x+5\\
    &\ln{x}+x^2=3
\end{align*}

\section{根的分析}
设$f(x)=\sin{x}-6x-5$, $g(x)=\ln{x}+x^2-3$.
那么有简单的不等式
\[\begin{array}{ll}
    f(-1)=\sin(-1)+1>0, & f(0)=-5<0;\\
    g(1)=-2<0, & g(2)=\ln{2}+1>0.
\end{array}\]
因此使用二分法求根时, $f$的根在$[-1,0]$中搜索, $g$的根在$[1,2]$中搜索.
使用Newton法与割线法迭代时, $f$使用$-1$作为起点, $g$使用$2$作为起点.

\section{算法}
以下列举二分法, Newton法, 割线法与使用Aitken加速的Newton法.

\begin{algorithm}[H]
    \caption{二分法}
    \KwIn{Function $f$, real number $a,b$ with $a<b$}
    \KwOut{Solution of $f(x)=0$}
    $a_0=a,b_0=b$\;
    \While{$b_n-a_n>1\times 10^{-8}$}{
        \eIf{$\sgn((b_n+a_n)/2)\sgn{a_n}<0$}{
            $a_{n+1}=a_n$\;
            $b_{n+1}=\dfrac{b_n+a_n}{2}$\;
            $n++$\;
        }{
            $a_{n+1}=\dfrac{b_n+a_n}{2}$\;
            $b_{n+1}=b_n$\;
            $n++$\;
        }
    }
    return $(a_n+b_n)/2$\;
\end{algorithm}

\begin{algorithm}[H]
    \caption{Newton法}
    \KwIn{Function $f$, real number $a$}
    \KwOut{Solution of $f(x)=0$}
    $a_0=a$\;
    \While{$|a_n-a_{n-1}|>1\times 10^{-8}$}{
        $a_{n+1}=a_n-\dfrac{f(a_n)}{f'(a_n)}$\;
        $n++$\;
    }
    return $a_n$\;
\end{algorithm}

\begin{algorithm}[H]
    \caption{割线法}
    \KwIn{Function $f$, real number $a,b$ with $a<x_0<b$, where $f(x_0)=0$}
    \KwOut{Solution of $f(x)=0$, i.e. $x_0$}
    $a_0=a,a_1=b$\;
    \While{$|a_n-a_{n-1}|>1\times 10^{-8}$}{
        $a_{n+1}=a_n-\dfrac{(a_n-a_{n-1})f(a_{n-1})}{f(a_n)-f(a_{n-1})}$\;
        $n++$\;
    }
    return $a_n$\;
\end{algorithm}

\begin{algorithm}[H]
    \caption{使用Aitken加速的Newton法}
    \KwIn{Function $f$, real number $a$}
    \KwOut{Solution of $f(x)=0$}
    Define function $\varphi(x)=x-\dfrac{f(x)}{f'(x)}$, $\Phi(x)=\dfrac{x\varphi(\varphi(x))-\varphi^2(x)}{x-2\varphi(x)+\varphi(\varphi(x))}$\;
    \While{$|a_n-a_{n-1}|>1\times 10^{-8}$}{
        $a_{n+1}=\Phi(a_n)$\;
        $n++$\;
    }
    return $a_n$\;
\end{algorithm}

\section{运行结果}
程序源代码分别参考文件夹中的\verb|program2-1.c|至\verb|program2-4.c|.
\begin{table}[H]
    \begin{tabular}{|c|c|c|c|c|}
    \hline
     & 二分法 & Newton法 & 割线法 & Aitken加速的Newton法 \\ \hline
    $f$的零点 & $-0.97089892$ & $-0.97089892$ & $-0.97089892$ & $-0.97089892$ \\ \hline
    $g$的零点 & $1.59214294$ & $1.59214295$ & $1.59214294$ & $1.59214294$ \\ \hline
    $f$的迭代次数 & $27$ & $2$ & $4$ & $32$ \\ \hline
    $g$的迭代次数 & $27$ & $3$ & $6$ & $102$ \\ \hline
    \end{tabular}
    \caption{结果图表}
\end{table}

可以得到迭代效率Newton法$>$割线法$>$二分法$>$Aitken加速的Newton法.
使用Aitken加速后反而迭代效率降低的原因尚不清楚.

\section{运行环境}
\noindent 程序运行环境: Dell Inspiron 14 Plus 7420, Linux 5.15.146.1-microsoft-standard-WSL2.\\
编译器: gcc version 9.4.0 (Ubuntu 9.4.0-Ubuntu 20.04.2)

\end{document}