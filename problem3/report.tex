\documentclass{article}

\usepackage{ctex}
\usepackage{geometry}
\usepackage{amsmath}
\usepackage{amssymb}
\usepackage{amsthm}
\usepackage[ruled,linesnumbered]{algorithm2e}
\usepackage{float}

\title{计算方法第三次上机作业}
\author{曾梦辰\thanks{学号: 202011999050}}
\date{\today}

\begin{document}
\maketitle

\section{问题}
使用追赶法求解线性方程$AX=Y$的解, 其中
\begin{itemize}
    \item $A$为三对角矩阵
    \[A=\begin{bmatrix}
        b_1 & c_1 & 0 & \cdots & 0 & 0 & 0 \\
        a_2 & b_2 & c_2 & \cdots & 0 & 0 & 0 \\
        0 & a_3 & b_3 & \cdots & 0 & 0 & 0 \\
        \vdots & \vdots & \vdots & \ddots & \vdots & \vdots & \vdots \\
        0 & 0 & 0 & \cdots & a_{n-1} & b_{n-1} & c_{n-1} \\
        0 & 0 & 0 & \cdots & 0 & a_n & b_n
    \end{bmatrix}\]
    \item $A$为周期三对角矩阵
    \[A=\begin{bmatrix}
        b_1 & c_1 & 0 & \cdots & 0 & 0 & a_1 \\
        a_2 & b_2 & c_2 & \cdots & 0 & 0 & 0 \\
        0 & a_3 & b_3 & \cdots & 0 & 0 & 0 \\
        \vdots & \vdots & \vdots & \ddots & \vdots & \vdots & \vdots \\
        0 & 0 & 0 & \cdots & a_{n-1} & b_{n-1} & c_{n-1} \\
        c_n & 0 & 0 & \cdots & 0 & a_n & b_n
    \end{bmatrix}\]
\end{itemize}
要求采用矩阵稀疏存储的技巧, 并将求解过程写成函数形式.

\section{算法}

追赶法本质为Gauss消元法, 我们列举伪代码如下:

\begin{algorithm}[H]
    \caption{追赶法}
    \KwIn{Tridiagonal matrix $A$, column vector $Y$}
    \KwOut{Solution of $AX=Y$}
    $i=2$\;
    \While{$i\leq n$}{
        $b_i=b_i-a_{i}c_{i-1}/b_{i-1}$\;
        $y_i=y_i-a_{i}y_{i-1}/b_{i-1}$\;
        $i++$\;
    }
    $x_n=y_n/b_n$\;
    $i=n-1$\;
    \While{$i>0$}{
        $x_i=(y_i-c_ix_{i+1})/b_i$\;
        $i--$\;
    }
    return $X=(x_1,\cdots,x_n)^\top$\;
\end{algorithm}

对于周期三对角矩阵, 我们作分解
\[A=\begin{bmatrix}
    b_1 & c_1 & 0 & \cdots & 0 & 0 & 0 \\
    a_2 & b_2 & c_2 & \cdots & 0 & 0 & 0 \\
    \vdots & \vdots & \vdots & \ddots & \vdots & \vdots & \vdots \\
    0 & 0 & 0 & \cdots & a_{n-2} & b_{n-2} & c_{n-2} \\
    0 & 0 & 0 & \cdots & 0 & a_{n-1} & b_{n-1}
\end{bmatrix}\begin{bmatrix}
    u_1 \\ u_2 \\ \vdots \\ u_{n-2} \\ u_{n-1}
\end{bmatrix}=\begin{bmatrix}
    -a_1 \\ 0 \\ \vdots \\ 0 \\ -c_{n-1}
\end{bmatrix}\]
及
\[A=\begin{bmatrix}
    b_1 & c_1 & 0 & \cdots & 0 & 0 & 0 \\
    a_2 & b_2 & c_2 & \cdots & 0 & 0 & 0 \\
    \vdots & \vdots & \vdots & \ddots & \vdots & \vdots & \vdots \\
    0 & 0 & 0 & \cdots & a_{n-2} & b_{n-2} & c_{n-2} \\
    0 & 0 & 0 & \cdots & 0 & a_{n-1} & b_{n-1}
\end{bmatrix}\begin{bmatrix}
    v_1 \\ v_2 \\ \vdots \\ v_{n-2} \\ v_{n-1}
\end{bmatrix}=\begin{bmatrix}
    y_1 \\ y_2 \\ \vdots \\ y_{n-2} \\ y_{n-1}
\end{bmatrix}\]
使用追赶法解出$U$和$V$, 那么可以验证
\[x_n=\frac{y_n-c_nv_1-a_nv_{n-1}}{b_n+c_nu_1+a_nu_{n-1}}\]
及
\[x_i=u_ix_n+v_i,\ i=1,2,\cdots,n-1\]
给出了解向量$X$.

\section{运行结果}
我们使用两个工作示例:
\[\begin{bmatrix}
    1 & 1 \\
    2 & 1
\end{bmatrix}X=\begin{bmatrix}
    2 \\ 1
\end{bmatrix}\]
代码\verb|program3-1.c|编译的程序给出正确结果$(-1,3)^\top$;
\[\begin{bmatrix}
    1 & 1 & 1 \\
    0 & 2 & 0 \\
    1 & 0 & 3
\end{bmatrix}X=\begin{bmatrix}
    1 \\ 1 \\ 0
\end{bmatrix}\]
代码\verb|program3-2.c|编译的程序给出正确结果$\left(\dfrac{3}{4},\dfrac{1}{2},-\dfrac{1}{4}\right)^\top$.

\section{运行环境}
\noindent 程序运行环境: Dell Inspiron 14 Plus 7420, Linux 5.15.146.1-microsoft-standard-WSL2.\\
编译器: gcc version 9.4.0 (Ubuntu 9.4.0-Ubuntu 20.04.2)

\end{document}