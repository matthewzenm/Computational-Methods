\documentclass{ctexart}

\usepackage{amsmath}
\usepackage{amsthm}
\usepackage{amssymb}
\usepackage{float}
\usepackage{geometry}
\usepackage[shortlabels]{enumitem}
\usepackage{hyperref}
\usepackage[ruled,linesnumbered]{algorithm2e}

\title{计算方法第一次上机作业}
\author{曾梦辰\thanks{学号: 202011999050}}
\date{\today}

\begin{document}
\maketitle
\begin{abstract}
    本次作业完成教材第20页17题, 分析舍入误差与有效数.
\end{abstract}

\section{问题}
设$\displaystyle S_N=\sum_{j=2}^N\frac{1}{j^2-1}$, 其精确值为$\displaystyle\frac{1}{2}\left(\frac{3}{2}-\frac{1}{N}-\frac{1}{N+1}\right)$.
\begin{enumerate}[(1)]
    \item 编写按从大到小 (求和指标上升) 的顺序计算$S_N$的通用程序, 即
    \[S_N=\frac{1}{2^2-1}+\frac{1}{3^2-1}+\cdots+\frac{1}{N^2-1}\]
    \item 编写按从小到大 (求和指标下降) 的顺序计算$S_N$的通用程序, 即
    \[S_N=\frac{1}{N^2-1}+\frac{1}{(N-1)^2-1}+\cdots+\frac{1}{2^2-1}\]
    \item 按以上程序计算$S_{10^2},S_{10^4},S_{10^6}$, 并指出有效位数, 数据使用单精度格式储存.
    分析以上结果.
\end{enumerate}
\section{算法}
我们分别使用以下算法进行计算.
\begin{algorithm}
    \caption{从大到小相加}
    \KwIn{Natural number $N$}
    \KwOut{$\sum_{2\leq j\leq N}\frac{1}{j^2-1}$}
    Define $j=2,S=0$\;
    \For{$j\leq N$}{
        $S+=\frac{1}{j^2-1}$\;
        $j+=1$}
    return $S$
\end{algorithm}

\begin{algorithm}
    \caption{从小到大相加}
    \KwIn{Natural number $N$}
    \KwOut{$\sum_{2\leq j\leq N}\frac{1}{j^2-1}$}
    Define $j=N,S=0$\;
    \For{$j\geq 2$}{
        $S+=\frac{1}{j^2-1}$\;
        $j-=1$}
    return $S$
\end{algorithm}


\section{运行结果}
程序源代码参考文件夹中的\verb|program.c|, 以下列举运行结果.
\begin{table}[H]
    \centering
    \begin{tabular}{|c|c|c|c|}
    \hline
    $N$    & 精确值        & 从大到小         & 从小到大         \\ \hline
    $10^2$ & $0.74004950$ & $0.74004948$   & $0.74004954$   \\ \hline
    $10^4$ & $0.74990000$ & $0.74985212$   & $0.74989998$   \\ \hline
    $10^6$ & $0.74990000$ & $0.74985212$   & $0.74999899$   \\ \hline
    \end{tabular}
    \caption{运行结果}
\end{table}

使用$\varepsilon_1(N)$表示从大到小计算时的绝对误差限, $\varepsilon_2(N)$表示从小到大计算时的绝对误差限.
当$N=10^2$时, 从大到小计算有$7$位有效数, 从小到大计算有$7$位有效数, 且$\varepsilon_1(10^2)<\varepsilon_2(10^2)$;
当$N=10^4$时, 从大到小计算有$3$位有效数, 从小到大计算有$3$位有效数, 且$\varepsilon_1(10^4)>\varepsilon_2(10^4)$;
当$N=10^6$时, 从大到小计算有$3$为有效数, 从小到大计算有$4$位有效数, 自然$\varepsilon_1(10^6)>\varepsilon_2(10^6)$.

\section{结果分析}
通过以上结果可以得到两点:
\begin{enumerate}
    \item 随着$N$增大, 两种算法的有效位数均在下降.
    这是因为随着$N$增大, 计算量增大, 舍入误差逐渐累积导致绝对误差限增大.
    \item 从大到小计算的有效数位数总是不大于从小到大计算的有效数位数.
    这是因为计算时, 为了执行大数加小数的加法, 计算机会对齐大数与小数的有效数, 导致大数加小数时在计算后期大量损失有效数, 从而产生更大的误差.
\end{enumerate}


\section{运行环境}
\noindent 程序运行环境: Microsoft Corporation Surface Pro 6, Windows 11 23H2.\\
编译器: gcc version 13.2.0 (Rev3, Built by MSYS2 project).

\end{document}